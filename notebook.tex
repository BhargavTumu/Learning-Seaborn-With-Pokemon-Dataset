
% Default to the notebook output style

    


% Inherit from the specified cell style.




    
\documentclass[11pt]{article}

    
    
    \usepackage[T1]{fontenc}
    % Nicer default font (+ math font) than Computer Modern for most use cases
    \usepackage{mathpazo}

    % Basic figure setup, for now with no caption control since it's done
    % automatically by Pandoc (which extracts ![](path) syntax from Markdown).
    \usepackage{graphicx}
    % We will generate all images so they have a width \maxwidth. This means
    % that they will get their normal width if they fit onto the page, but
    % are scaled down if they would overflow the margins.
    \makeatletter
    \def\maxwidth{\ifdim\Gin@nat@width>\linewidth\linewidth
    \else\Gin@nat@width\fi}
    \makeatother
    \let\Oldincludegraphics\includegraphics
    % Set max figure width to be 80% of text width, for now hardcoded.
    \renewcommand{\includegraphics}[1]{\Oldincludegraphics[width=.8\maxwidth]{#1}}
    % Ensure that by default, figures have no caption (until we provide a
    % proper Figure object with a Caption API and a way to capture that
    % in the conversion process - todo).
    \usepackage{caption}
    \DeclareCaptionLabelFormat{nolabel}{}
    \captionsetup{labelformat=nolabel}

    \usepackage{adjustbox} % Used to constrain images to a maximum size 
    \usepackage{xcolor} % Allow colors to be defined
    \usepackage{enumerate} % Needed for markdown enumerations to work
    \usepackage{geometry} % Used to adjust the document margins
    \usepackage{amsmath} % Equations
    \usepackage{amssymb} % Equations
    \usepackage{textcomp} % defines textquotesingle
    % Hack from http://tex.stackexchange.com/a/47451/13684:
    \AtBeginDocument{%
        \def\PYZsq{\textquotesingle}% Upright quotes in Pygmentized code
    }
    \usepackage{upquote} % Upright quotes for verbatim code
    \usepackage{eurosym} % defines \euro
    \usepackage[mathletters]{ucs} % Extended unicode (utf-8) support
    \usepackage[utf8x]{inputenc} % Allow utf-8 characters in the tex document
    \usepackage{fancyvrb} % verbatim replacement that allows latex
    \usepackage{grffile} % extends the file name processing of package graphics 
                         % to support a larger range 
    % The hyperref package gives us a pdf with properly built
    % internal navigation ('pdf bookmarks' for the table of contents,
    % internal cross-reference links, web links for URLs, etc.)
    \usepackage{hyperref}
    \usepackage{longtable} % longtable support required by pandoc >1.10
    \usepackage{booktabs}  % table support for pandoc > 1.12.2
    \usepackage[inline]{enumitem} % IRkernel/repr support (it uses the enumerate* environment)
    \usepackage[normalem]{ulem} % ulem is needed to support strikethroughs (\sout)
                                % normalem makes italics be italics, not underlines
    

    
    
    % Colors for the hyperref package
    \definecolor{urlcolor}{rgb}{0,.145,.698}
    \definecolor{linkcolor}{rgb}{.71,0.21,0.01}
    \definecolor{citecolor}{rgb}{.12,.54,.11}

    % ANSI colors
    \definecolor{ansi-black}{HTML}{3E424D}
    \definecolor{ansi-black-intense}{HTML}{282C36}
    \definecolor{ansi-red}{HTML}{E75C58}
    \definecolor{ansi-red-intense}{HTML}{B22B31}
    \definecolor{ansi-green}{HTML}{00A250}
    \definecolor{ansi-green-intense}{HTML}{007427}
    \definecolor{ansi-yellow}{HTML}{DDB62B}
    \definecolor{ansi-yellow-intense}{HTML}{B27D12}
    \definecolor{ansi-blue}{HTML}{208FFB}
    \definecolor{ansi-blue-intense}{HTML}{0065CA}
    \definecolor{ansi-magenta}{HTML}{D160C4}
    \definecolor{ansi-magenta-intense}{HTML}{A03196}
    \definecolor{ansi-cyan}{HTML}{60C6C8}
    \definecolor{ansi-cyan-intense}{HTML}{258F8F}
    \definecolor{ansi-white}{HTML}{C5C1B4}
    \definecolor{ansi-white-intense}{HTML}{A1A6B2}

    % commands and environments needed by pandoc snippets
    % extracted from the output of `pandoc -s`
    \providecommand{\tightlist}{%
      \setlength{\itemsep}{0pt}\setlength{\parskip}{0pt}}
    \DefineVerbatimEnvironment{Highlighting}{Verbatim}{commandchars=\\\{\}}
    % Add ',fontsize=\small' for more characters per line
    \newenvironment{Shaded}{}{}
    \newcommand{\KeywordTok}[1]{\textcolor[rgb]{0.00,0.44,0.13}{\textbf{{#1}}}}
    \newcommand{\DataTypeTok}[1]{\textcolor[rgb]{0.56,0.13,0.00}{{#1}}}
    \newcommand{\DecValTok}[1]{\textcolor[rgb]{0.25,0.63,0.44}{{#1}}}
    \newcommand{\BaseNTok}[1]{\textcolor[rgb]{0.25,0.63,0.44}{{#1}}}
    \newcommand{\FloatTok}[1]{\textcolor[rgb]{0.25,0.63,0.44}{{#1}}}
    \newcommand{\CharTok}[1]{\textcolor[rgb]{0.25,0.44,0.63}{{#1}}}
    \newcommand{\StringTok}[1]{\textcolor[rgb]{0.25,0.44,0.63}{{#1}}}
    \newcommand{\CommentTok}[1]{\textcolor[rgb]{0.38,0.63,0.69}{\textit{{#1}}}}
    \newcommand{\OtherTok}[1]{\textcolor[rgb]{0.00,0.44,0.13}{{#1}}}
    \newcommand{\AlertTok}[1]{\textcolor[rgb]{1.00,0.00,0.00}{\textbf{{#1}}}}
    \newcommand{\FunctionTok}[1]{\textcolor[rgb]{0.02,0.16,0.49}{{#1}}}
    \newcommand{\RegionMarkerTok}[1]{{#1}}
    \newcommand{\ErrorTok}[1]{\textcolor[rgb]{1.00,0.00,0.00}{\textbf{{#1}}}}
    \newcommand{\NormalTok}[1]{{#1}}
    
    % Additional commands for more recent versions of Pandoc
    \newcommand{\ConstantTok}[1]{\textcolor[rgb]{0.53,0.00,0.00}{{#1}}}
    \newcommand{\SpecialCharTok}[1]{\textcolor[rgb]{0.25,0.44,0.63}{{#1}}}
    \newcommand{\VerbatimStringTok}[1]{\textcolor[rgb]{0.25,0.44,0.63}{{#1}}}
    \newcommand{\SpecialStringTok}[1]{\textcolor[rgb]{0.73,0.40,0.53}{{#1}}}
    \newcommand{\ImportTok}[1]{{#1}}
    \newcommand{\DocumentationTok}[1]{\textcolor[rgb]{0.73,0.13,0.13}{\textit{{#1}}}}
    \newcommand{\AnnotationTok}[1]{\textcolor[rgb]{0.38,0.63,0.69}{\textbf{\textit{{#1}}}}}
    \newcommand{\CommentVarTok}[1]{\textcolor[rgb]{0.38,0.63,0.69}{\textbf{\textit{{#1}}}}}
    \newcommand{\VariableTok}[1]{\textcolor[rgb]{0.10,0.09,0.49}{{#1}}}
    \newcommand{\ControlFlowTok}[1]{\textcolor[rgb]{0.00,0.44,0.13}{\textbf{{#1}}}}
    \newcommand{\OperatorTok}[1]{\textcolor[rgb]{0.40,0.40,0.40}{{#1}}}
    \newcommand{\BuiltInTok}[1]{{#1}}
    \newcommand{\ExtensionTok}[1]{{#1}}
    \newcommand{\PreprocessorTok}[1]{\textcolor[rgb]{0.74,0.48,0.00}{{#1}}}
    \newcommand{\AttributeTok}[1]{\textcolor[rgb]{0.49,0.56,0.16}{{#1}}}
    \newcommand{\InformationTok}[1]{\textcolor[rgb]{0.38,0.63,0.69}{\textbf{\textit{{#1}}}}}
    \newcommand{\WarningTok}[1]{\textcolor[rgb]{0.38,0.63,0.69}{\textbf{\textit{{#1}}}}}
    
    
    % Define a nice break command that doesn't care if a line doesn't already
    % exist.
    \def\br{\hspace*{\fill} \\* }
    % Math Jax compatability definitions
    \def\gt{>}
    \def\lt{<}
    % Document parameters
    \title{Learning Seaborn with Pokemon}
    
    
    

    % Pygments definitions
    
\makeatletter
\def\PY@reset{\let\PY@it=\relax \let\PY@bf=\relax%
    \let\PY@ul=\relax \let\PY@tc=\relax%
    \let\PY@bc=\relax \let\PY@ff=\relax}
\def\PY@tok#1{\csname PY@tok@#1\endcsname}
\def\PY@toks#1+{\ifx\relax#1\empty\else%
    \PY@tok{#1}\expandafter\PY@toks\fi}
\def\PY@do#1{\PY@bc{\PY@tc{\PY@ul{%
    \PY@it{\PY@bf{\PY@ff{#1}}}}}}}
\def\PY#1#2{\PY@reset\PY@toks#1+\relax+\PY@do{#2}}

\expandafter\def\csname PY@tok@w\endcsname{\def\PY@tc##1{\textcolor[rgb]{0.73,0.73,0.73}{##1}}}
\expandafter\def\csname PY@tok@c\endcsname{\let\PY@it=\textit\def\PY@tc##1{\textcolor[rgb]{0.25,0.50,0.50}{##1}}}
\expandafter\def\csname PY@tok@cp\endcsname{\def\PY@tc##1{\textcolor[rgb]{0.74,0.48,0.00}{##1}}}
\expandafter\def\csname PY@tok@k\endcsname{\let\PY@bf=\textbf\def\PY@tc##1{\textcolor[rgb]{0.00,0.50,0.00}{##1}}}
\expandafter\def\csname PY@tok@kp\endcsname{\def\PY@tc##1{\textcolor[rgb]{0.00,0.50,0.00}{##1}}}
\expandafter\def\csname PY@tok@kt\endcsname{\def\PY@tc##1{\textcolor[rgb]{0.69,0.00,0.25}{##1}}}
\expandafter\def\csname PY@tok@o\endcsname{\def\PY@tc##1{\textcolor[rgb]{0.40,0.40,0.40}{##1}}}
\expandafter\def\csname PY@tok@ow\endcsname{\let\PY@bf=\textbf\def\PY@tc##1{\textcolor[rgb]{0.67,0.13,1.00}{##1}}}
\expandafter\def\csname PY@tok@nb\endcsname{\def\PY@tc##1{\textcolor[rgb]{0.00,0.50,0.00}{##1}}}
\expandafter\def\csname PY@tok@nf\endcsname{\def\PY@tc##1{\textcolor[rgb]{0.00,0.00,1.00}{##1}}}
\expandafter\def\csname PY@tok@nc\endcsname{\let\PY@bf=\textbf\def\PY@tc##1{\textcolor[rgb]{0.00,0.00,1.00}{##1}}}
\expandafter\def\csname PY@tok@nn\endcsname{\let\PY@bf=\textbf\def\PY@tc##1{\textcolor[rgb]{0.00,0.00,1.00}{##1}}}
\expandafter\def\csname PY@tok@ne\endcsname{\let\PY@bf=\textbf\def\PY@tc##1{\textcolor[rgb]{0.82,0.25,0.23}{##1}}}
\expandafter\def\csname PY@tok@nv\endcsname{\def\PY@tc##1{\textcolor[rgb]{0.10,0.09,0.49}{##1}}}
\expandafter\def\csname PY@tok@no\endcsname{\def\PY@tc##1{\textcolor[rgb]{0.53,0.00,0.00}{##1}}}
\expandafter\def\csname PY@tok@nl\endcsname{\def\PY@tc##1{\textcolor[rgb]{0.63,0.63,0.00}{##1}}}
\expandafter\def\csname PY@tok@ni\endcsname{\let\PY@bf=\textbf\def\PY@tc##1{\textcolor[rgb]{0.60,0.60,0.60}{##1}}}
\expandafter\def\csname PY@tok@na\endcsname{\def\PY@tc##1{\textcolor[rgb]{0.49,0.56,0.16}{##1}}}
\expandafter\def\csname PY@tok@nt\endcsname{\let\PY@bf=\textbf\def\PY@tc##1{\textcolor[rgb]{0.00,0.50,0.00}{##1}}}
\expandafter\def\csname PY@tok@nd\endcsname{\def\PY@tc##1{\textcolor[rgb]{0.67,0.13,1.00}{##1}}}
\expandafter\def\csname PY@tok@s\endcsname{\def\PY@tc##1{\textcolor[rgb]{0.73,0.13,0.13}{##1}}}
\expandafter\def\csname PY@tok@sd\endcsname{\let\PY@it=\textit\def\PY@tc##1{\textcolor[rgb]{0.73,0.13,0.13}{##1}}}
\expandafter\def\csname PY@tok@si\endcsname{\let\PY@bf=\textbf\def\PY@tc##1{\textcolor[rgb]{0.73,0.40,0.53}{##1}}}
\expandafter\def\csname PY@tok@se\endcsname{\let\PY@bf=\textbf\def\PY@tc##1{\textcolor[rgb]{0.73,0.40,0.13}{##1}}}
\expandafter\def\csname PY@tok@sr\endcsname{\def\PY@tc##1{\textcolor[rgb]{0.73,0.40,0.53}{##1}}}
\expandafter\def\csname PY@tok@ss\endcsname{\def\PY@tc##1{\textcolor[rgb]{0.10,0.09,0.49}{##1}}}
\expandafter\def\csname PY@tok@sx\endcsname{\def\PY@tc##1{\textcolor[rgb]{0.00,0.50,0.00}{##1}}}
\expandafter\def\csname PY@tok@m\endcsname{\def\PY@tc##1{\textcolor[rgb]{0.40,0.40,0.40}{##1}}}
\expandafter\def\csname PY@tok@gh\endcsname{\let\PY@bf=\textbf\def\PY@tc##1{\textcolor[rgb]{0.00,0.00,0.50}{##1}}}
\expandafter\def\csname PY@tok@gu\endcsname{\let\PY@bf=\textbf\def\PY@tc##1{\textcolor[rgb]{0.50,0.00,0.50}{##1}}}
\expandafter\def\csname PY@tok@gd\endcsname{\def\PY@tc##1{\textcolor[rgb]{0.63,0.00,0.00}{##1}}}
\expandafter\def\csname PY@tok@gi\endcsname{\def\PY@tc##1{\textcolor[rgb]{0.00,0.63,0.00}{##1}}}
\expandafter\def\csname PY@tok@gr\endcsname{\def\PY@tc##1{\textcolor[rgb]{1.00,0.00,0.00}{##1}}}
\expandafter\def\csname PY@tok@ge\endcsname{\let\PY@it=\textit}
\expandafter\def\csname PY@tok@gs\endcsname{\let\PY@bf=\textbf}
\expandafter\def\csname PY@tok@gp\endcsname{\let\PY@bf=\textbf\def\PY@tc##1{\textcolor[rgb]{0.00,0.00,0.50}{##1}}}
\expandafter\def\csname PY@tok@go\endcsname{\def\PY@tc##1{\textcolor[rgb]{0.53,0.53,0.53}{##1}}}
\expandafter\def\csname PY@tok@gt\endcsname{\def\PY@tc##1{\textcolor[rgb]{0.00,0.27,0.87}{##1}}}
\expandafter\def\csname PY@tok@err\endcsname{\def\PY@bc##1{\setlength{\fboxsep}{0pt}\fcolorbox[rgb]{1.00,0.00,0.00}{1,1,1}{\strut ##1}}}
\expandafter\def\csname PY@tok@kc\endcsname{\let\PY@bf=\textbf\def\PY@tc##1{\textcolor[rgb]{0.00,0.50,0.00}{##1}}}
\expandafter\def\csname PY@tok@kd\endcsname{\let\PY@bf=\textbf\def\PY@tc##1{\textcolor[rgb]{0.00,0.50,0.00}{##1}}}
\expandafter\def\csname PY@tok@kn\endcsname{\let\PY@bf=\textbf\def\PY@tc##1{\textcolor[rgb]{0.00,0.50,0.00}{##1}}}
\expandafter\def\csname PY@tok@kr\endcsname{\let\PY@bf=\textbf\def\PY@tc##1{\textcolor[rgb]{0.00,0.50,0.00}{##1}}}
\expandafter\def\csname PY@tok@bp\endcsname{\def\PY@tc##1{\textcolor[rgb]{0.00,0.50,0.00}{##1}}}
\expandafter\def\csname PY@tok@fm\endcsname{\def\PY@tc##1{\textcolor[rgb]{0.00,0.00,1.00}{##1}}}
\expandafter\def\csname PY@tok@vc\endcsname{\def\PY@tc##1{\textcolor[rgb]{0.10,0.09,0.49}{##1}}}
\expandafter\def\csname PY@tok@vg\endcsname{\def\PY@tc##1{\textcolor[rgb]{0.10,0.09,0.49}{##1}}}
\expandafter\def\csname PY@tok@vi\endcsname{\def\PY@tc##1{\textcolor[rgb]{0.10,0.09,0.49}{##1}}}
\expandafter\def\csname PY@tok@vm\endcsname{\def\PY@tc##1{\textcolor[rgb]{0.10,0.09,0.49}{##1}}}
\expandafter\def\csname PY@tok@sa\endcsname{\def\PY@tc##1{\textcolor[rgb]{0.73,0.13,0.13}{##1}}}
\expandafter\def\csname PY@tok@sb\endcsname{\def\PY@tc##1{\textcolor[rgb]{0.73,0.13,0.13}{##1}}}
\expandafter\def\csname PY@tok@sc\endcsname{\def\PY@tc##1{\textcolor[rgb]{0.73,0.13,0.13}{##1}}}
\expandafter\def\csname PY@tok@dl\endcsname{\def\PY@tc##1{\textcolor[rgb]{0.73,0.13,0.13}{##1}}}
\expandafter\def\csname PY@tok@s2\endcsname{\def\PY@tc##1{\textcolor[rgb]{0.73,0.13,0.13}{##1}}}
\expandafter\def\csname PY@tok@sh\endcsname{\def\PY@tc##1{\textcolor[rgb]{0.73,0.13,0.13}{##1}}}
\expandafter\def\csname PY@tok@s1\endcsname{\def\PY@tc##1{\textcolor[rgb]{0.73,0.13,0.13}{##1}}}
\expandafter\def\csname PY@tok@mb\endcsname{\def\PY@tc##1{\textcolor[rgb]{0.40,0.40,0.40}{##1}}}
\expandafter\def\csname PY@tok@mf\endcsname{\def\PY@tc##1{\textcolor[rgb]{0.40,0.40,0.40}{##1}}}
\expandafter\def\csname PY@tok@mh\endcsname{\def\PY@tc##1{\textcolor[rgb]{0.40,0.40,0.40}{##1}}}
\expandafter\def\csname PY@tok@mi\endcsname{\def\PY@tc##1{\textcolor[rgb]{0.40,0.40,0.40}{##1}}}
\expandafter\def\csname PY@tok@il\endcsname{\def\PY@tc##1{\textcolor[rgb]{0.40,0.40,0.40}{##1}}}
\expandafter\def\csname PY@tok@mo\endcsname{\def\PY@tc##1{\textcolor[rgb]{0.40,0.40,0.40}{##1}}}
\expandafter\def\csname PY@tok@ch\endcsname{\let\PY@it=\textit\def\PY@tc##1{\textcolor[rgb]{0.25,0.50,0.50}{##1}}}
\expandafter\def\csname PY@tok@cm\endcsname{\let\PY@it=\textit\def\PY@tc##1{\textcolor[rgb]{0.25,0.50,0.50}{##1}}}
\expandafter\def\csname PY@tok@cpf\endcsname{\let\PY@it=\textit\def\PY@tc##1{\textcolor[rgb]{0.25,0.50,0.50}{##1}}}
\expandafter\def\csname PY@tok@c1\endcsname{\let\PY@it=\textit\def\PY@tc##1{\textcolor[rgb]{0.25,0.50,0.50}{##1}}}
\expandafter\def\csname PY@tok@cs\endcsname{\let\PY@it=\textit\def\PY@tc##1{\textcolor[rgb]{0.25,0.50,0.50}{##1}}}

\def\PYZbs{\char`\\}
\def\PYZus{\char`\_}
\def\PYZob{\char`\{}
\def\PYZcb{\char`\}}
\def\PYZca{\char`\^}
\def\PYZam{\char`\&}
\def\PYZlt{\char`\<}
\def\PYZgt{\char`\>}
\def\PYZsh{\char`\#}
\def\PYZpc{\char`\%}
\def\PYZdl{\char`\$}
\def\PYZhy{\char`\-}
\def\PYZsq{\char`\'}
\def\PYZdq{\char`\"}
\def\PYZti{\char`\~}
% for compatibility with earlier versions
\def\PYZat{@}
\def\PYZlb{[}
\def\PYZrb{]}
\makeatother


    % Exact colors from NB
    \definecolor{incolor}{rgb}{0.0, 0.0, 0.5}
    \definecolor{outcolor}{rgb}{0.545, 0.0, 0.0}



    
    % Prevent overflowing lines due to hard-to-break entities
    \sloppy 
    % Setup hyperref package
    \hypersetup{
      breaklinks=true,  % so long urls are correctly broken across lines
      colorlinks=true,
      urlcolor=urlcolor,
      linkcolor=linkcolor,
      citecolor=citecolor,
      }
    % Slightly bigger margins than the latex defaults
    
    \geometry{verbose,tmargin=1in,bmargin=1in,lmargin=1in,rmargin=1in}
    
    

    \begin{document}
    
    
    \maketitle
    
    

    
    \begin{Verbatim}[commandchars=\\\{\}]
{\color{incolor}In [{\color{incolor}6}]:} \PY{c+c1}{\PYZsh{}Importing required libraries}
        
        \PY{k+kn}{import} \PY{n+nn}{pandas} \PY{k}{as} \PY{n+nn}{pd}
        \PY{k+kn}{from} \PY{n+nn}{matplotlib} \PY{k}{import} \PY{n}{pyplot} \PY{k}{as} \PY{n}{plt}
        \PY{o}{\PYZpc{}}\PY{k}{matplotlib} inline
        
        \PY{k+kn}{import} \PY{n+nn}{seaborn} \PY{k}{as} \PY{n+nn}{sns}
\end{Verbatim}


    \begin{Verbatim}[commandchars=\\\{\}]
{\color{incolor}In [{\color{incolor}7}]:} \PY{c+c1}{\PYZsh{}Import the pokemon dataset into a pandas dataframe. The dataset csv file should be in the same folder as the notebbok.}
        
        \PY{n}{dataset} \PY{o}{=} \PY{n}{pd}\PY{o}{.}\PY{n}{read\PYZus{}csv}\PY{p}{(}\PY{l+s+s1}{\PYZsq{}}\PY{l+s+s1}{Pokemon.csv}\PY{l+s+s1}{\PYZsq{}}\PY{p}{,} \PY{n}{index\PYZus{}col}\PY{o}{=}\PY{l+m+mi}{0}\PY{p}{)} 
        \PY{c+c1}{\PYZsh{} The index\PYZus{}col = 0 simply means that we will treat the first column in the dataset as the ID column.}
\end{Verbatim}


    \begin{Verbatim}[commandchars=\\\{\}]
{\color{incolor}In [{\color{incolor}8}]:} \PY{n}{dataset}\PY{o}{.}\PY{n}{head}\PY{p}{(}\PY{p}{)}
\end{Verbatim}


\begin{Verbatim}[commandchars=\\\{\}]
{\color{outcolor}Out[{\color{outcolor}8}]:}          Name Type 1  Type 2  Total  HP  Attack  Defense  Sp. Atk  Sp. Def  \textbackslash{}
        \#                                                                            
        1   Bulbasaur  Grass  Poison    318  45      49       49       65       65   
        2     Ivysaur  Grass  Poison    405  60      62       63       80       80   
        3    Venusaur  Grass  Poison    525  80      82       83      100      100   
        4  Charmander   Fire     NaN    309  39      52       43       60       50   
        5  Charmeleon   Fire     NaN    405  58      64       58       80       65   
        
           Speed  Stage  Legendary  
        \#                           
        1     45      1      False  
        2     60      2      False  
        3     80      3      False  
        4     65      1      False  
        5     80      2      False  
\end{Verbatim}
            
    \begin{Verbatim}[commandchars=\\\{\}]
{\color{incolor}In [{\color{incolor}23}]:} \PY{c+c1}{\PYZsh{} 1. Default Scatter Plot }
         
         \PY{n}{sns}\PY{o}{.}\PY{n}{set\PYZus{}style}\PY{p}{(}\PY{n}{style}\PY{o}{=}\PY{l+s+s2}{\PYZdq{}}\PY{l+s+s2}{whitegrid}\PY{l+s+s2}{\PYZdq{}}\PY{p}{)} \PY{c+c1}{\PYZsh{} Produces the background grid. Other options are }
                                          \PY{c+c1}{\PYZsh{} darkgrid, dark, white, ticks. }
         
         \PY{n}{sns}\PY{o}{.}\PY{n}{lmplot}\PY{p}{(}\PY{n}{x}\PY{o}{=}\PY{l+s+s1}{\PYZsq{}}\PY{l+s+s1}{Attack}\PY{l+s+s1}{\PYZsq{}}\PY{p}{,}\PY{n}{y}\PY{o}{=}\PY{l+s+s1}{\PYZsq{}}\PY{l+s+s1}{Defense}\PY{l+s+s1}{\PYZsq{}}\PY{p}{,}\PY{n}{data}\PY{o}{=}\PY{n}{dataset}\PY{p}{,}\PY{n}{size}\PY{o}{=}\PY{l+m+mi}{8}\PY{p}{)}
\end{Verbatim}


\begin{Verbatim}[commandchars=\\\{\}]
{\color{outcolor}Out[{\color{outcolor}23}]:} <seaborn.axisgrid.FacetGrid at 0x1ccefcffbe0>
\end{Verbatim}
            
    \begin{center}
    \adjustimage{max size={0.9\linewidth}{0.9\paperheight}}{output_3_1.png}
    \end{center}
    { \hspace*{\fill} \\}
    
    \begin{Verbatim}[commandchars=\\\{\}]
{\color{incolor}In [{\color{incolor}17}]:} \PY{c+c1}{\PYZsh{} Scatterplot Arguments}
         \PY{n}{sns}\PY{o}{.}\PY{n}{lmplot}\PY{p}{(}\PY{n}{x}\PY{o}{=}\PY{l+s+s1}{\PYZsq{}}\PY{l+s+s1}{Attack}\PY{l+s+s1}{\PYZsq{}}\PY{p}{,} \PY{n}{y}\PY{o}{=}\PY{l+s+s1}{\PYZsq{}}\PY{l+s+s1}{Defense}\PY{l+s+s1}{\PYZsq{}}\PY{p}{,} \PY{n}{data}\PY{o}{=}\PY{n}{dataset}\PY{p}{,} 
                    \PY{n}{fit\PYZus{}reg}\PY{o}{=}\PY{k+kc}{False}\PY{p}{,} \PY{c+c1}{\PYZsh{} No regression line}
                    \PY{n}{hue}\PY{o}{=}\PY{l+s+s1}{\PYZsq{}}\PY{l+s+s1}{Stage}\PY{l+s+s1}{\PYZsq{}}\PY{p}{,} \PY{c+c1}{\PYZsh{} Color by evolution stage}
                    \PY{n}{size}\PY{o}{=}\PY{l+m+mi}{8}\PY{p}{)}
\end{Verbatim}


\begin{Verbatim}[commandchars=\\\{\}]
{\color{outcolor}Out[{\color{outcolor}17}]:} <seaborn.axisgrid.FacetGrid at 0x1ccef500898>
\end{Verbatim}
            
    \begin{center}
    \adjustimage{max size={0.9\linewidth}{0.9\paperheight}}{output_4_1.png}
    \end{center}
    { \hspace*{\fill} \\}
    
    \begin{Verbatim}[commandchars=\\\{\}]
{\color{incolor}In [{\color{incolor}22}]:} \PY{l+s+sd}{\PYZsq{}\PYZsq{}\PYZsq{}}
         \PY{l+s+sd}{As we can see in thr above chart , we don\PYZsq{}t have axis limits, the x and y axis go below 0. To set}
         \PY{l+s+sd}{proper axis limit we use matplotlib.}
         
         \PY{l+s+sd}{Remember, Seaborn is a high\PYZhy{}level interface to Matplotlib. From our experience, Seaborn will get you most of the way there, but you\PYZsq{}ll sometimes need to bring in Matplotlib.}
         
         \PY{l+s+sd}{Setting your axes limits is one of those times, but the process is pretty simple:}
         
         \PY{l+s+sd}{First, invoke your Seaborn plotting function as normal.}
         
         \PY{l+s+sd}{Then, invoke Matplotlib\PYZsq{}s customization functions. In this case, we\PYZsq{}ll use its }
         \PY{l+s+sd}{ylim() and xlim() functions.}
         
         \PY{l+s+sd}{For more information about matplotlib plotting , see the documentation here :}
         \PY{l+s+sd}{https://matplotlib.org/users/pyplot\PYZus{}tutorial.html}
         
         \PY{l+s+sd}{\PYZsq{}\PYZsq{}\PYZsq{}}
         
         \PY{n}{sns}\PY{o}{.}\PY{n}{lmplot}\PY{p}{(}\PY{n}{x}\PY{o}{=}\PY{l+s+s1}{\PYZsq{}}\PY{l+s+s1}{Attack}\PY{l+s+s1}{\PYZsq{}}\PY{p}{,} \PY{n}{y}\PY{o}{=}\PY{l+s+s1}{\PYZsq{}}\PY{l+s+s1}{Defense}\PY{l+s+s1}{\PYZsq{}}\PY{p}{,} \PY{n}{data}\PY{o}{=}\PY{n}{dataset}\PY{p}{,} \PY{n}{fit\PYZus{}reg}\PY{o}{=}\PY{k+kc}{False}\PY{p}{,} \PY{n}{hue}\PY{o}{=}\PY{l+s+s1}{\PYZsq{}}\PY{l+s+s1}{Stage}\PY{l+s+s1}{\PYZsq{}}\PY{p}{,} \PY{n}{size}\PY{o}{=}\PY{l+m+mi}{8}\PY{p}{)}
         \PY{n}{plt}\PY{o}{.}\PY{n}{xlim}\PY{p}{(}\PY{l+m+mi}{0}\PY{p}{,} \PY{k+kc}{None}\PY{p}{)} \PY{c+c1}{\PYZsh{} (min, max)}
         \PY{n}{plt}\PY{o}{.}\PY{n}{ylim}\PY{p}{(}\PY{l+m+mi}{0}\PY{p}{,} \PY{k+kc}{None}\PY{p}{)}
\end{Verbatim}


\begin{Verbatim}[commandchars=\\\{\}]
{\color{outcolor}Out[{\color{outcolor}22}]:} (0, 189.7475432989191)
\end{Verbatim}
            
    \begin{center}
    \adjustimage{max size={0.9\linewidth}{0.9\paperheight}}{output_5_1.png}
    \end{center}
    { \hspace*{\fill} \\}
    
    \begin{Verbatim}[commandchars=\\\{\}]
{\color{incolor}In [{\color{incolor}26}]:} \PY{c+c1}{\PYZsh{} BoxPlot}
         
         \PY{l+s+sd}{\PYZsq{}\PYZsq{}\PYZsq{}}
         \PY{l+s+sd}{How to read boxplots :}
         \PY{l+s+sd}{https://www.statisticshowto.datasciencecentral.com/probability\PYZhy{}and\PYZhy{}statistics/descriptive\PYZhy{}statistics/box\PYZhy{}plot/}
         
         
         \PY{l+s+sd}{\PYZsq{}\PYZsq{}\PYZsq{}}
         
         \PY{n}{plt}\PY{o}{.}\PY{n}{figure}\PY{p}{(}\PY{n}{figsize}\PY{o}{=}\PY{p}{(}\PY{l+m+mi}{16}\PY{p}{,}\PY{l+m+mi}{10}\PY{p}{)}\PY{p}{)} \PY{c+c1}{\PYZsh{} To set the size of the boxplot, boxplot doesn\PYZsq{}t have a size argument.}
         \PY{n}{sns}\PY{o}{.}\PY{n}{boxplot}\PY{p}{(}\PY{n}{data}\PY{o}{=}\PY{n}{dataset}\PY{p}{)}
\end{Verbatim}


\begin{Verbatim}[commandchars=\\\{\}]
{\color{outcolor}Out[{\color{outcolor}26}]:} <matplotlib.axes.\_subplots.AxesSubplot at 0x1ccf1c8dfd0>
\end{Verbatim}
            
    \begin{center}
    \adjustimage{max size={0.9\linewidth}{0.9\paperheight}}{output_6_1.png}
    \end{center}
    { \hspace*{\fill} \\}
    
    \begin{Verbatim}[commandchars=\\\{\}]
{\color{incolor}In [{\color{incolor}28}]:} \PY{l+s+sd}{\PYZdq{}\PYZdq{}\PYZdq{}}
         \PY{l+s+sd}{Well, that\PYZsq{}s a reasonable start, but there are some columns we\PYZsq{}d probably like to remove:}
         
         \PY{l+s+sd}{We can remove the Total since we have individual stats.}
         \PY{l+s+sd}{We can remove the Stage and Legendary columns because they aren\PYZsq{}t combat stats.}
         \PY{l+s+sd}{In turns out that this isn\PYZsq{}t easy to do within Seaborn alone. Instead, it\PYZsq{}s much simpler to pre\PYZhy{}format your DataFrame.}
         
         \PY{l+s+sd}{Let\PYZsq{}s create a new DataFrame called stats\PYZus{}df that only keeps the stats columns:}
         \PY{l+s+sd}{\PYZdq{}\PYZdq{}\PYZdq{}}
         
         \PY{n}{stats\PYZus{}df} \PY{o}{=} \PY{n}{dataset}\PY{o}{.}\PY{n}{drop}\PY{p}{(}\PY{p}{[}\PY{l+s+s1}{\PYZsq{}}\PY{l+s+s1}{Total}\PY{l+s+s1}{\PYZsq{}}\PY{p}{,} \PY{l+s+s1}{\PYZsq{}}\PY{l+s+s1}{Stage}\PY{l+s+s1}{\PYZsq{}}\PY{p}{,} \PY{l+s+s1}{\PYZsq{}}\PY{l+s+s1}{Legendary}\PY{l+s+s1}{\PYZsq{}}\PY{p}{]}\PY{p}{,} \PY{n}{axis}\PY{o}{=}\PY{l+m+mi}{1}\PY{p}{)}
         
         \PY{c+c1}{\PYZsh{} Pandas cheatsheet : http://pandas.pydata.org/pandas\PYZhy{}docs/stable/getting\PYZus{}started/10min.html}
         
         \PY{n}{plt}\PY{o}{.}\PY{n}{figure}\PY{p}{(}\PY{n}{figsize}\PY{o}{=}\PY{p}{(}\PY{l+m+mi}{16}\PY{p}{,}\PY{l+m+mi}{10}\PY{p}{)}\PY{p}{)}
         \PY{n}{sns}\PY{o}{.}\PY{n}{boxplot}\PY{p}{(}\PY{n}{data}\PY{o}{=}\PY{n}{stats\PYZus{}df}\PY{p}{)}
\end{Verbatim}


\begin{Verbatim}[commandchars=\\\{\}]
{\color{outcolor}Out[{\color{outcolor}28}]:} <matplotlib.axes.\_subplots.AxesSubplot at 0x1ccf1bfa208>
\end{Verbatim}
            
    \begin{center}
    \adjustimage{max size={0.9\linewidth}{0.9\paperheight}}{output_7_1.png}
    \end{center}
    { \hspace*{\fill} \\}
    
    \begin{Verbatim}[commandchars=\\\{\}]
{\color{incolor}In [{\color{incolor}31}]:} \PY{c+c1}{\PYZsh{} ViolinPlot }
         \PY{c+c1}{\PYZsh{} https://mode.com/blog/violin\PYZhy{}plot\PYZhy{}examples}
         
         \PY{n}{plt}\PY{o}{.}\PY{n}{figure}\PY{p}{(}\PY{n}{figsize}\PY{o}{=}\PY{p}{(}\PY{l+m+mi}{16}\PY{p}{,}\PY{l+m+mi}{10}\PY{p}{)}\PY{p}{)}
         \PY{n}{sns}\PY{o}{.}\PY{n}{violinplot}\PY{p}{(}\PY{n}{x}\PY{o}{=}\PY{l+s+s1}{\PYZsq{}}\PY{l+s+s1}{Type 1}\PY{l+s+s1}{\PYZsq{}}\PY{p}{,} \PY{n}{y}\PY{o}{=}\PY{l+s+s1}{\PYZsq{}}\PY{l+s+s1}{Attack}\PY{l+s+s1}{\PYZsq{}}\PY{p}{,} \PY{n}{data}\PY{o}{=}\PY{n}{dataset}\PY{p}{)}
\end{Verbatim}


\begin{Verbatim}[commandchars=\\\{\}]
{\color{outcolor}Out[{\color{outcolor}31}]:} <matplotlib.axes.\_subplots.AxesSubplot at 0x1ccefbe85f8>
\end{Verbatim}
            
    \begin{center}
    \adjustimage{max size={0.9\linewidth}{0.9\paperheight}}{output_8_1.png}
    \end{center}
    { \hspace*{\fill} \\}
    
    \begin{Verbatim}[commandchars=\\\{\}]
{\color{incolor}In [{\color{incolor}32}]:} \PY{c+c1}{\PYZsh{} Setting custom colors for different types}
         \PY{n}{pkmn\PYZus{}type\PYZus{}colors} \PY{o}{=} \PY{p}{[}\PY{l+s+s1}{\PYZsq{}}\PY{l+s+s1}{\PYZsh{}78C850}\PY{l+s+s1}{\PYZsq{}}\PY{p}{,}  \PY{c+c1}{\PYZsh{} Grass}
                             \PY{l+s+s1}{\PYZsq{}}\PY{l+s+s1}{\PYZsh{}F08030}\PY{l+s+s1}{\PYZsq{}}\PY{p}{,}  \PY{c+c1}{\PYZsh{} Fire}
                             \PY{l+s+s1}{\PYZsq{}}\PY{l+s+s1}{\PYZsh{}6890F0}\PY{l+s+s1}{\PYZsq{}}\PY{p}{,}  \PY{c+c1}{\PYZsh{} Water}
                             \PY{l+s+s1}{\PYZsq{}}\PY{l+s+s1}{\PYZsh{}A8B820}\PY{l+s+s1}{\PYZsq{}}\PY{p}{,}  \PY{c+c1}{\PYZsh{} Bug}
                             \PY{l+s+s1}{\PYZsq{}}\PY{l+s+s1}{\PYZsh{}A8A878}\PY{l+s+s1}{\PYZsq{}}\PY{p}{,}  \PY{c+c1}{\PYZsh{} Normal}
                             \PY{l+s+s1}{\PYZsq{}}\PY{l+s+s1}{\PYZsh{}A040A0}\PY{l+s+s1}{\PYZsq{}}\PY{p}{,}  \PY{c+c1}{\PYZsh{} Poison}
                             \PY{l+s+s1}{\PYZsq{}}\PY{l+s+s1}{\PYZsh{}F8D030}\PY{l+s+s1}{\PYZsq{}}\PY{p}{,}  \PY{c+c1}{\PYZsh{} Electric}
                             \PY{l+s+s1}{\PYZsq{}}\PY{l+s+s1}{\PYZsh{}E0C068}\PY{l+s+s1}{\PYZsq{}}\PY{p}{,}  \PY{c+c1}{\PYZsh{} Ground}
                             \PY{l+s+s1}{\PYZsq{}}\PY{l+s+s1}{\PYZsh{}EE99AC}\PY{l+s+s1}{\PYZsq{}}\PY{p}{,}  \PY{c+c1}{\PYZsh{} Fairy}
                             \PY{l+s+s1}{\PYZsq{}}\PY{l+s+s1}{\PYZsh{}C03028}\PY{l+s+s1}{\PYZsq{}}\PY{p}{,}  \PY{c+c1}{\PYZsh{} Fighting}
                             \PY{l+s+s1}{\PYZsq{}}\PY{l+s+s1}{\PYZsh{}F85888}\PY{l+s+s1}{\PYZsq{}}\PY{p}{,}  \PY{c+c1}{\PYZsh{} Psychic}
                             \PY{l+s+s1}{\PYZsq{}}\PY{l+s+s1}{\PYZsh{}B8A038}\PY{l+s+s1}{\PYZsq{}}\PY{p}{,}  \PY{c+c1}{\PYZsh{} Rock}
                             \PY{l+s+s1}{\PYZsq{}}\PY{l+s+s1}{\PYZsh{}705898}\PY{l+s+s1}{\PYZsq{}}\PY{p}{,}  \PY{c+c1}{\PYZsh{} Ghost}
                             \PY{l+s+s1}{\PYZsq{}}\PY{l+s+s1}{\PYZsh{}98D8D8}\PY{l+s+s1}{\PYZsq{}}\PY{p}{,}  \PY{c+c1}{\PYZsh{} Ice}
                             \PY{l+s+s1}{\PYZsq{}}\PY{l+s+s1}{\PYZsh{}7038F8}\PY{l+s+s1}{\PYZsq{}}\PY{p}{,}  \PY{c+c1}{\PYZsh{} Dragon}
                            \PY{p}{]}
         
         \PY{n}{plt}\PY{o}{.}\PY{n}{figure}\PY{p}{(}\PY{n}{figsize}\PY{o}{=}\PY{p}{(}\PY{l+m+mi}{16}\PY{p}{,}\PY{l+m+mi}{10}\PY{p}{)}\PY{p}{)}
         \PY{n}{sns}\PY{o}{.}\PY{n}{violinplot}\PY{p}{(}\PY{n}{x}\PY{o}{=}\PY{l+s+s1}{\PYZsq{}}\PY{l+s+s1}{Type 1}\PY{l+s+s1}{\PYZsq{}}\PY{p}{,} \PY{n}{y}\PY{o}{=}\PY{l+s+s1}{\PYZsq{}}\PY{l+s+s1}{Attack}\PY{l+s+s1}{\PYZsq{}}\PY{p}{,} \PY{n}{data}\PY{o}{=}\PY{n}{dataset}\PY{p}{,} \PY{n}{palette}\PY{o}{=}\PY{n}{pkmn\PYZus{}type\PYZus{}colors}\PY{p}{)}
\end{Verbatim}


\begin{Verbatim}[commandchars=\\\{\}]
{\color{outcolor}Out[{\color{outcolor}32}]:} <matplotlib.axes.\_subplots.AxesSubplot at 0x1ccf0059b38>
\end{Verbatim}
            
    \begin{center}
    \adjustimage{max size={0.9\linewidth}{0.9\paperheight}}{output_9_1.png}
    \end{center}
    { \hspace*{\fill} \\}
    
    \begin{Verbatim}[commandchars=\\\{\}]
{\color{incolor}In [{\color{incolor}39}]:} \PY{c+c1}{\PYZsh{} SwarmPlot}
         \PY{c+c1}{\PYZsh{} This visualization will show each point, while \PYZdq{}stacking\PYZdq{} those with similar values:}
         
         \PY{n}{plt}\PY{o}{.}\PY{n}{figure}\PY{p}{(}\PY{n}{figsize}\PY{o}{=}\PY{p}{(}\PY{l+m+mi}{16}\PY{p}{,}\PY{l+m+mi}{9}\PY{p}{)}\PY{p}{)}
         \PY{n}{sns}\PY{o}{.}\PY{n}{swarmplot}\PY{p}{(}\PY{n}{x}\PY{o}{=}\PY{l+s+s1}{\PYZsq{}}\PY{l+s+s1}{Type 1}\PY{l+s+s1}{\PYZsq{}}\PY{p}{,} \PY{n}{y}\PY{o}{=}\PY{l+s+s1}{\PYZsq{}}\PY{l+s+s1}{Attack}\PY{l+s+s1}{\PYZsq{}}\PY{p}{,} \PY{n}{data}\PY{o}{=}\PY{n}{dataset}\PY{p}{,} \PY{n}{palette}\PY{o}{=}\PY{n}{pkmn\PYZus{}type\PYZus{}colors}\PY{p}{,} \PY{n}{size}\PY{o}{=}\PY{l+m+mi}{10}\PY{p}{)}
\end{Verbatim}


\begin{Verbatim}[commandchars=\\\{\}]
{\color{outcolor}Out[{\color{outcolor}39}]:} <matplotlib.axes.\_subplots.AxesSubplot at 0x1ccf1f9f7f0>
\end{Verbatim}
            
    \begin{center}
    \adjustimage{max size={0.9\linewidth}{0.9\paperheight}}{output_10_1.png}
    \end{center}
    { \hspace*{\fill} \\}
    
    \begin{Verbatim}[commandchars=\\\{\}]
{\color{incolor}In [{\color{incolor}47}]:} \PY{c+c1}{\PYZsh{} Overlaying Plots}
         \PY{l+s+sd}{\PYZsq{}\PYZsq{}\PYZsq{}}
         \PY{l+s+sd}{Combining Violin and Swarm Plots.}
         
         \PY{l+s+sd}{It\PYZsq{}s pretty straightforward to overlay plots using Seaborn, and it works the same way as }
         \PY{l+s+sd}{with Matplotlib. Here\PYZsq{}s what we\PYZsq{}ll do:}
         
         \PY{l+s+sd}{First, we\PYZsq{}ll make our figure larger using Matplotlib.}
         \PY{l+s+sd}{Then, we\PYZsq{}ll plot the violin plot. However, we\PYZsq{}ll set inner=None to remove the bars inside the violins.}
         \PY{l+s+sd}{Next, we\PYZsq{}ll plot the swarm plot. This time, we\PYZsq{}ll make the points black so they pop out more.}
         \PY{l+s+sd}{Finally, we\PYZsq{}ll set a title using Matplotlib.}
         
         \PY{l+s+sd}{\PYZsq{}\PYZsq{}\PYZsq{}}
         
         \PY{c+c1}{\PYZsh{} Set figure size with matplotlib}
         \PY{n}{plt}\PY{o}{.}\PY{n}{figure}\PY{p}{(}\PY{n}{figsize}\PY{o}{=}\PY{p}{(}\PY{l+m+mi}{16}\PY{p}{,}\PY{l+m+mi}{9}\PY{p}{)}\PY{p}{)}
         
         \PY{c+c1}{\PYZsh{}Violin Plot}
         
         \PY{n}{sns}\PY{o}{.}\PY{n}{violinplot}\PY{p}{(}\PY{n}{x}\PY{o}{=}\PY{l+s+s1}{\PYZsq{}}\PY{l+s+s1}{Type 1}\PY{l+s+s1}{\PYZsq{}}\PY{p}{,} \PY{n}{y}\PY{o}{=}\PY{l+s+s1}{\PYZsq{}}\PY{l+s+s1}{Attack}\PY{l+s+s1}{\PYZsq{}}\PY{p}{,} \PY{n}{data}\PY{o}{=}\PY{n}{dataset}\PY{p}{,}
                       \PY{n}{inner}\PY{o}{=}\PY{k+kc}{None}\PY{p}{,} \PY{c+c1}{\PYZsh{} Removes the bars inside the violins}
                       \PY{n}{palette}\PY{o}{=}\PY{n}{pkmn\PYZus{}type\PYZus{}colors}\PY{p}{,}
                       \PY{n}{size} \PY{o}{=} \PY{l+m+mi}{10}\PY{p}{)}
         
         \PY{c+c1}{\PYZsh{} SwarmPlot}
         
         \PY{n}{sns}\PY{o}{.}\PY{n}{swarmplot}\PY{p}{(}\PY{n}{x}\PY{o}{=}\PY{l+s+s1}{\PYZsq{}}\PY{l+s+s1}{Type 1}\PY{l+s+s1}{\PYZsq{}}\PY{p}{,} \PY{n}{y}\PY{o}{=}\PY{l+s+s1}{\PYZsq{}}\PY{l+s+s1}{Attack}\PY{l+s+s1}{\PYZsq{}}\PY{p}{,} \PY{n}{data}\PY{o}{=}\PY{n}{dataset}\PY{p}{,} 
                       \PY{n}{color}\PY{o}{=}\PY{l+s+s1}{\PYZsq{}}\PY{l+s+s1}{k}\PY{l+s+s1}{\PYZsq{}}\PY{p}{,} \PY{c+c1}{\PYZsh{}Make points black}
                      \PY{n}{alpha}\PY{o}{=}\PY{l+m+mf}{0.7} \PY{p}{)} \PY{c+c1}{\PYZsh{} make them slightly transperent}
         
         \PY{c+c1}{\PYZsh{} Set title }
         \PY{n}{plt}\PY{o}{.}\PY{n}{title}\PY{p}{(}\PY{l+s+s1}{\PYZsq{}}\PY{l+s+s1}{Attack by Type}\PY{l+s+s1}{\PYZsq{}}\PY{p}{)}
\end{Verbatim}


\begin{Verbatim}[commandchars=\\\{\}]
{\color{outcolor}Out[{\color{outcolor}47}]:} Text(0.5, 1.0, 'Attack by Type')
\end{Verbatim}
            
    \begin{center}
    \adjustimage{max size={0.9\linewidth}{0.9\paperheight}}{output_11_1.png}
    \end{center}
    { \hspace*{\fill} \\}
    
    \begin{Verbatim}[commandchars=\\\{\}]
{\color{incolor}In [{\color{incolor}48}]:} \PY{c+c1}{\PYZsh{} COMBINING ALL THE STATS INTO ONE PLOT}
         
         \PY{l+s+sd}{\PYZsq{}\PYZsq{}\PYZsq{}}
         \PY{l+s+sd}{To put all the stats into one plot first we have to do some data wrangling.}
         \PY{l+s+sd}{\PYZsq{}\PYZsq{}\PYZsq{}}
         
         \PY{c+c1}{\PYZsh{} Current data}
         \PY{n}{stats\PYZus{}df}\PY{o}{.}\PY{n}{head}\PY{p}{(}\PY{p}{)}
\end{Verbatim}


\begin{Verbatim}[commandchars=\\\{\}]
{\color{outcolor}Out[{\color{outcolor}48}]:}          Name Type 1  Type 2  HP  Attack  Defense  Sp. Atk  Sp. Def  Speed
         \#                                                                         
         1   Bulbasaur  Grass  Poison  45      49       49       65       65     45
         2     Ivysaur  Grass  Poison  60      62       63       80       80     60
         3    Venusaur  Grass  Poison  80      82       83      100      100     80
         4  Charmander   Fire     NaN  39      52       43       60       50     65
         5  Charmeleon   Fire     NaN  58      64       58       80       65     80
\end{Verbatim}
            
    \begin{Verbatim}[commandchars=\\\{\}]
{\color{incolor}In [{\color{incolor}49}]:} \PY{l+s+sd}{\PYZsq{}\PYZsq{}\PYZsq{}}
         \PY{l+s+sd}{As we can see , we have different columns for differnet stats. To show all the stats in one plot we have to }
         \PY{l+s+sd}{make the stats a key value pair, this can be done using melt function in pandas.}
         
         \PY{l+s+sd}{Pandas\PYZsq{}s melt() function takes 3 arguments:}
         
         \PY{l+s+sd}{First, the DataFrame to melt.}
         \PY{l+s+sd}{Second, ID variables to keep (Pandas will melt all of the other ones).}
         \PY{l+s+sd}{Finally, a name for the new, melted variable.}
         
         \PY{l+s+sd}{\PYZsq{}\PYZsq{}\PYZsq{}}
         
         \PY{n}{melted\PYZus{}df} \PY{o}{=} \PY{n}{pd}\PY{o}{.}\PY{n}{melt}\PY{p}{(}\PY{n}{stats\PYZus{}df}\PY{p}{,} 
                             \PY{n}{id\PYZus{}vars}\PY{o}{=}\PY{p}{[}\PY{l+s+s2}{\PYZdq{}}\PY{l+s+s2}{Name}\PY{l+s+s2}{\PYZdq{}}\PY{p}{,} \PY{l+s+s2}{\PYZdq{}}\PY{l+s+s2}{Type 1}\PY{l+s+s2}{\PYZdq{}}\PY{p}{,} \PY{l+s+s2}{\PYZdq{}}\PY{l+s+s2}{Type 2}\PY{l+s+s2}{\PYZdq{}}\PY{p}{]}\PY{p}{,} \PY{c+c1}{\PYZsh{} variables to keep}
                            \PY{n}{var\PYZus{}name} \PY{o}{=}\PY{l+s+s2}{\PYZdq{}}\PY{l+s+s2}{Stat}\PY{l+s+s2}{\PYZdq{}} \PY{c+c1}{\PYZsh{}Name of melted variable}
                            \PY{p}{)}
         \PY{n}{melted\PYZus{}df}\PY{o}{.}\PY{n}{head}\PY{p}{(}\PY{p}{)}
\end{Verbatim}


\begin{Verbatim}[commandchars=\\\{\}]
{\color{outcolor}Out[{\color{outcolor}49}]:}          Name Type 1  Type 2 Stat  value
         0   Bulbasaur  Grass  Poison   HP     45
         1     Ivysaur  Grass  Poison   HP     60
         2    Venusaur  Grass  Poison   HP     80
         3  Charmander   Fire     NaN   HP     39
         4  Charmeleon   Fire     NaN   HP     58
\end{Verbatim}
            
    \begin{Verbatim}[commandchars=\\\{\}]
{\color{incolor}In [{\color{incolor}50}]:} \PY{l+s+sd}{\PYZsq{}\PYZsq{}\PYZsq{}}
         \PY{l+s+sd}{All 6 of the stat columns have been \PYZdq{}melted\PYZdq{} into one, and the new Stat column indicates the original stat (HP, Attack, Defense, Sp. Attack, Sp. Defense, or Speed). For example, it\PYZsq{}s hard to see here, but Bulbasaur now has 6 rows of data.}
         
         \PY{l+s+sd}{In fact, if you print the shape of these two DataFrames...}
         \PY{l+s+sd}{\PYZsq{}\PYZsq{}\PYZsq{}}
         \PY{n+nb}{print}\PY{p}{(} \PY{n}{stats\PYZus{}df}\PY{o}{.}\PY{n}{shape} \PY{p}{)}
         \PY{n+nb}{print}\PY{p}{(} \PY{n}{melted\PYZus{}df}\PY{o}{.}\PY{n}{shape} \PY{p}{)}
\end{Verbatim}


    \begin{Verbatim}[commandchars=\\\{\}]
(151, 9)
(906, 5)

    \end{Verbatim}

    \begin{Verbatim}[commandchars=\\\{\}]
{\color{incolor}In [{\color{incolor}53}]:} \PY{c+c1}{\PYZsh{}Plotting all the stats using swarmplot}
         
         \PY{n}{plt}\PY{o}{.}\PY{n}{figure}\PY{p}{(}\PY{n}{figsize}\PY{o}{=}\PY{p}{(}\PY{l+m+mi}{16}\PY{p}{,}\PY{l+m+mi}{9}\PY{p}{)}\PY{p}{)}
         \PY{n}{sns}\PY{o}{.}\PY{n}{swarmplot}\PY{p}{(}\PY{n}{x}\PY{o}{=}\PY{l+s+s1}{\PYZsq{}}\PY{l+s+s1}{Stat}\PY{l+s+s1}{\PYZsq{}}\PY{p}{,} \PY{n}{y}\PY{o}{=}\PY{l+s+s1}{\PYZsq{}}\PY{l+s+s1}{value}\PY{l+s+s1}{\PYZsq{}}\PY{p}{,} \PY{n}{data}\PY{o}{=}\PY{n}{melted\PYZus{}df}\PY{p}{,} \PY{n}{hue}\PY{o}{=}\PY{l+s+s1}{\PYZsq{}}\PY{l+s+s1}{Type 1}\PY{l+s+s1}{\PYZsq{}} \PY{p}{,} \PY{n}{size}\PY{o}{=}\PY{l+m+mi}{10}\PY{p}{)}
\end{Verbatim}


\begin{Verbatim}[commandchars=\\\{\}]
{\color{outcolor}Out[{\color{outcolor}53}]:} <matplotlib.axes.\_subplots.AxesSubplot at 0x1ccf36a8a58>
\end{Verbatim}
            
    \begin{center}
    \adjustimage{max size={0.9\linewidth}{0.9\paperheight}}{output_15_1.png}
    \end{center}
    { \hspace*{\fill} \\}
    
    \begin{Verbatim}[commandchars=\\\{\}]
{\color{incolor}In [{\color{incolor}57}]:} \PY{c+c1}{\PYZsh{} Making the plot a little more readable}
         
         \PY{c+c1}{\PYZsh{} Enalarge the plot}
         
         \PY{n}{plt}\PY{o}{.}\PY{n}{figure}\PY{p}{(}\PY{n}{figsize}\PY{o}{=}\PY{p}{(}\PY{l+m+mi}{16}\PY{p}{,}\PY{l+m+mi}{9}\PY{p}{)}\PY{p}{)}
         
         \PY{c+c1}{\PYZsh{} SwarmPlot }
         
         \PY{n}{sns}\PY{o}{.}\PY{n}{swarmplot}\PY{p}{(}\PY{n}{x}\PY{o}{=}\PY{l+s+s1}{\PYZsq{}}\PY{l+s+s1}{Stat}\PY{l+s+s1}{\PYZsq{}}\PY{p}{,} \PY{n}{y}\PY{o}{=}\PY{l+s+s1}{\PYZsq{}}\PY{l+s+s1}{value}\PY{l+s+s1}{\PYZsq{}}\PY{p}{,} \PY{n}{data}\PY{o}{=}\PY{n}{melted\PYZus{}df}\PY{p}{,} \PY{n}{hue}\PY{o}{=}\PY{l+s+s1}{\PYZsq{}}\PY{l+s+s1}{Type 1}\PY{l+s+s1}{\PYZsq{}}\PY{p}{,} \PY{n}{split}\PY{o}{=}\PY{k+kc}{True}\PY{p}{,} \PY{n}{size}\PY{o}{=}\PY{l+m+mi}{7}\PY{p}{,} \PY{n}{palette}\PY{o}{=}\PY{n}{pkmn\PYZus{}type\PYZus{}colors}\PY{p}{)}
         
         \PY{c+c1}{\PYZsh{} Adjust the y\PYZhy{}axis}
         
         \PY{n}{plt}\PY{o}{.}\PY{n}{ylim}\PY{p}{(}\PY{l+m+mi}{0}\PY{p}{,}\PY{l+m+mi}{260}\PY{p}{)}
         
         \PY{c+c1}{\PYZsh{}Place legend to the right}
         \PY{n}{plt}\PY{o}{.}\PY{n}{legend}\PY{p}{(}\PY{n}{bbox\PYZus{}to\PYZus{}anchor}\PY{o}{=}\PY{p}{(}\PY{l+m+mi}{1}\PY{p}{,}\PY{l+m+mi}{1}\PY{p}{)}\PY{p}{,} \PY{n}{loc}\PY{o}{=}\PY{l+m+mi}{2}\PY{p}{)}
\end{Verbatim}


    \begin{Verbatim}[commandchars=\\\{\}]
C:\textbackslash{}Users\textbackslash{}bharg\textbackslash{}Anaconda3\textbackslash{}lib\textbackslash{}site-packages\textbackslash{}seaborn\textbackslash{}categorical.py:2783: UserWarning: The `split` parameter has been renamed to `dodge`.
  warnings.warn(msg, UserWarning)

    \end{Verbatim}

\begin{Verbatim}[commandchars=\\\{\}]
{\color{outcolor}Out[{\color{outcolor}57}]:} <matplotlib.legend.Legend at 0x1ccf51eb1d0>
\end{Verbatim}
            
    \begin{center}
    \adjustimage{max size={0.9\linewidth}{0.9\paperheight}}{output_16_2.png}
    \end{center}
    { \hspace*{\fill} \\}
    
    \begin{Verbatim}[commandchars=\\\{\}]
{\color{incolor}In [{\color{incolor}60}]:} \PY{c+c1}{\PYZsh{} HeatMap}
         
         \PY{c+c1}{\PYZsh{} calculate correlations}
         
         \PY{n}{corr} \PY{o}{=} \PY{n}{stats\PYZus{}df}\PY{o}{.}\PY{n}{corr}\PY{p}{(}\PY{p}{)}
         
         \PY{c+c1}{\PYZsh{}heatmap}
         \PY{n}{plt}\PY{o}{.}\PY{n}{figure}\PY{p}{(}\PY{n}{figsize}\PY{o}{=}\PY{p}{(}\PY{l+m+mi}{16}\PY{p}{,}\PY{l+m+mi}{9}\PY{p}{)}\PY{p}{)}
         \PY{n}{sns}\PY{o}{.}\PY{n}{heatmap}\PY{p}{(}\PY{n}{corr}\PY{p}{)}
\end{Verbatim}


\begin{Verbatim}[commandchars=\\\{\}]
{\color{outcolor}Out[{\color{outcolor}60}]:} <matplotlib.axes.\_subplots.AxesSubplot at 0x1ccf4d80320>
\end{Verbatim}
            
    \begin{center}
    \adjustimage{max size={0.9\linewidth}{0.9\paperheight}}{output_17_1.png}
    \end{center}
    { \hspace*{\fill} \\}
    
    \begin{Verbatim}[commandchars=\\\{\}]
{\color{incolor}In [{\color{incolor}61}]:} \PY{c+c1}{\PYZsh{}Histogram}
         \PY{n}{plt}\PY{o}{.}\PY{n}{figure}\PY{p}{(}\PY{n}{figsize}\PY{o}{=}\PY{p}{(}\PY{l+m+mi}{16}\PY{p}{,}\PY{l+m+mi}{9}\PY{p}{)}\PY{p}{)}
         \PY{n}{sns}\PY{o}{.}\PY{n}{distplot}\PY{p}{(}\PY{n}{dataset}\PY{o}{.}\PY{n}{Attack}\PY{p}{)}
\end{Verbatim}


    \begin{Verbatim}[commandchars=\\\{\}]
C:\textbackslash{}Users\textbackslash{}bharg\textbackslash{}Anaconda3\textbackslash{}lib\textbackslash{}site-packages\textbackslash{}matplotlib\textbackslash{}axes\textbackslash{}\_axes.py:6521: MatplotlibDeprecationWarning: 
The 'normed' kwarg was deprecated in Matplotlib 2.1 and will be removed in 3.1. Use 'density' instead.
  alternative="'density'", removal="3.1")

    \end{Verbatim}

\begin{Verbatim}[commandchars=\\\{\}]
{\color{outcolor}Out[{\color{outcolor}61}]:} <matplotlib.axes.\_subplots.AxesSubplot at 0x1ccf4d5dd30>
\end{Verbatim}
            
    \begin{center}
    \adjustimage{max size={0.9\linewidth}{0.9\paperheight}}{output_18_2.png}
    \end{center}
    { \hspace*{\fill} \\}
    
    \begin{Verbatim}[commandchars=\\\{\}]
{\color{incolor}In [{\color{incolor}71}]:} \PY{c+c1}{\PYZsh{} Bar\PYZhy{}plot :}
         \PY{c+c1}{\PYZsh{} Count Plot (a.k.a. Bar Plot)}
         \PY{n}{plt}\PY{o}{.}\PY{n}{figure}\PY{p}{(}\PY{n}{figsize}\PY{o}{=}\PY{p}{(}\PY{l+m+mi}{16}\PY{p}{,}\PY{l+m+mi}{9}\PY{p}{)}\PY{p}{)}
         \PY{n}{sns}\PY{o}{.}\PY{n}{countplot}\PY{p}{(}\PY{n}{x}\PY{o}{=}\PY{l+s+s1}{\PYZsq{}}\PY{l+s+s1}{Type 1}\PY{l+s+s1}{\PYZsq{}}\PY{p}{,} \PY{n}{data}\PY{o}{=}\PY{n}{dataset}\PY{p}{,} \PY{n}{palette}\PY{o}{=}\PY{n}{pkmn\PYZus{}type\PYZus{}colors}\PY{p}{)}
         
         \PY{c+c1}{\PYZsh{} Rotate x\PYZhy{}labels and increase size}
         \PY{n}{plt}\PY{o}{.}\PY{n}{xticks}\PY{p}{(}\PY{n}{rotation}\PY{o}{=}\PY{o}{\PYZhy{}}\PY{l+m+mi}{45}\PY{p}{,} \PY{n}{size}\PY{o}{=}\PY{l+m+mi}{16}\PY{p}{)}
         \PY{n}{plt}\PY{o}{.}\PY{n}{yticks}\PY{p}{(} \PY{n}{size}\PY{o}{=}\PY{l+m+mi}{16}\PY{p}{)}
\end{Verbatim}


\begin{Verbatim}[commandchars=\\\{\}]
{\color{outcolor}Out[{\color{outcolor}71}]:} (array([ 0.,  5., 10., 15., 20., 25., 30.]),
          <a list of 7 Text yticklabel objects>)
\end{Verbatim}
            
    \begin{center}
    \adjustimage{max size={0.9\linewidth}{0.9\paperheight}}{output_19_1.png}
    \end{center}
    { \hspace*{\fill} \\}
    
    \begin{Verbatim}[commandchars=\\\{\}]
{\color{incolor}In [{\color{incolor}74}]:} \PY{c+c1}{\PYZsh{} Factor plot  :Factor plots make it easy to separate plots by categorical classes.}
         
         \PY{n}{g} \PY{o}{=} \PY{n}{sns}\PY{o}{.}\PY{n}{factorplot}\PY{p}{(}\PY{n}{x}\PY{o}{=}\PY{l+s+s1}{\PYZsq{}}\PY{l+s+s1}{Type 1}\PY{l+s+s1}{\PYZsq{}}\PY{p}{,}
                           \PY{n}{y}\PY{o}{=}\PY{l+s+s1}{\PYZsq{}}\PY{l+s+s1}{Attack}\PY{l+s+s1}{\PYZsq{}}\PY{p}{,}
                           \PY{n}{data} \PY{o}{=} \PY{n}{dataset}\PY{p}{,}
                           \PY{n}{hue}\PY{o}{=}\PY{l+s+s1}{\PYZsq{}}\PY{l+s+s1}{Stage}\PY{l+s+s1}{\PYZsq{}}\PY{p}{,} \PY{c+c1}{\PYZsh{} Color by stage}
                           \PY{n}{col}\PY{o}{=}\PY{l+s+s1}{\PYZsq{}}\PY{l+s+s1}{Stage}\PY{l+s+s1}{\PYZsq{}}\PY{p}{,} \PY{c+c1}{\PYZsh{} Separate by stage}
                           \PY{n}{kind}\PY{o}{=}\PY{l+s+s1}{\PYZsq{}}\PY{l+s+s1}{swarm}\PY{l+s+s1}{\PYZsq{}}\PY{p}{)} \PY{c+c1}{\PYZsh{} Swarmplot}
         
         \PY{n}{g}\PY{o}{.}\PY{n}{set\PYZus{}xticklabels}\PY{p}{(}\PY{n}{rotation}\PY{o}{=}\PY{o}{\PYZhy{}}\PY{l+m+mi}{45}\PY{p}{)}
\end{Verbatim}


\begin{Verbatim}[commandchars=\\\{\}]
{\color{outcolor}Out[{\color{outcolor}74}]:} <seaborn.axisgrid.FacetGrid at 0x1ccf7431550>
\end{Verbatim}
            
    \begin{center}
    \adjustimage{max size={0.9\linewidth}{0.9\paperheight}}{output_20_1.png}
    \end{center}
    { \hspace*{\fill} \\}
    
    \begin{Verbatim}[commandchars=\\\{\}]
{\color{incolor}In [{\color{incolor}78}]:} \PY{c+c1}{\PYZsh{} Density Plot : Density plots display the distribution between two variables.}
         \PY{n}{plt}\PY{o}{.}\PY{n}{figure}\PY{p}{(}\PY{n}{figsize}\PY{o}{=}\PY{p}{(}\PY{l+m+mi}{16}\PY{p}{,}\PY{l+m+mi}{9}\PY{p}{)}\PY{p}{)}
         \PY{n}{sns}\PY{o}{.}\PY{n}{kdeplot}\PY{p}{(}\PY{n}{dataset}\PY{o}{.}\PY{n}{Attack}\PY{p}{,} \PY{n}{dataset}\PY{o}{.}\PY{n}{Defense}\PY{p}{)}
         \PY{n}{plt}\PY{o}{.}\PY{n}{xticks}\PY{p}{(}\PY{n}{size}\PY{o}{=}\PY{l+m+mi}{16}\PY{p}{)}
         \PY{n}{plt}\PY{o}{.}\PY{n}{yticks}\PY{p}{(} \PY{n}{size}\PY{o}{=}\PY{l+m+mi}{16}\PY{p}{)}
\end{Verbatim}


\begin{Verbatim}[commandchars=\\\{\}]
{\color{outcolor}Out[{\color{outcolor}78}]:} (array([-50.,   0.,  50., 100., 150., 200., 250.]),
          <a list of 7 Text yticklabel objects>)
\end{Verbatim}
            
    \begin{center}
    \adjustimage{max size={0.9\linewidth}{0.9\paperheight}}{output_21_1.png}
    \end{center}
    { \hspace*{\fill} \\}
    
    \begin{Verbatim}[commandchars=\\\{\}]
{\color{incolor}In [{\color{incolor}81}]:} \PY{c+c1}{\PYZsh{} Joint Distribution Plot : Joint distribution plots combine information from scatter plots and histograms to give you }
         \PY{c+c1}{\PYZsh{} detailed information for bi\PYZhy{}variate distributions.}
         
         \PY{n}{sns}\PY{o}{.}\PY{n}{jointplot}\PY{p}{(}\PY{n}{x}\PY{o}{=}\PY{l+s+s1}{\PYZsq{}}\PY{l+s+s1}{Attack}\PY{l+s+s1}{\PYZsq{}}\PY{p}{,} \PY{n}{y}\PY{o}{=}\PY{l+s+s1}{\PYZsq{}}\PY{l+s+s1}{Defense}\PY{l+s+s1}{\PYZsq{}}\PY{p}{,} \PY{n}{data}\PY{o}{=}\PY{n}{dataset}\PY{p}{)}
         
         \PY{n}{plt}\PY{o}{.}\PY{n}{figure}\PY{p}{(}\PY{n}{figsize}\PY{o}{=}\PY{p}{(}\PY{l+m+mi}{16}\PY{p}{,}\PY{l+m+mi}{9}\PY{p}{)}\PY{p}{)}
\end{Verbatim}


    \begin{Verbatim}[commandchars=\\\{\}]
C:\textbackslash{}Users\textbackslash{}bharg\textbackslash{}Anaconda3\textbackslash{}lib\textbackslash{}site-packages\textbackslash{}matplotlib\textbackslash{}axes\textbackslash{}\_axes.py:6521: MatplotlibDeprecationWarning: 
The 'normed' kwarg was deprecated in Matplotlib 2.1 and will be removed in 3.1. Use 'density' instead.
  alternative="'density'", removal="3.1")
C:\textbackslash{}Users\textbackslash{}bharg\textbackslash{}Anaconda3\textbackslash{}lib\textbackslash{}site-packages\textbackslash{}matplotlib\textbackslash{}axes\textbackslash{}\_axes.py:6521: MatplotlibDeprecationWarning: 
The 'normed' kwarg was deprecated in Matplotlib 2.1 and will be removed in 3.1. Use 'density' instead.
  alternative="'density'", removal="3.1")

    \end{Verbatim}

\begin{Verbatim}[commandchars=\\\{\}]
{\color{outcolor}Out[{\color{outcolor}81}]:} <Figure size 1152x648 with 0 Axes>
\end{Verbatim}
            
    \begin{center}
    \adjustimage{max size={0.9\linewidth}{0.9\paperheight}}{output_22_2.png}
    \end{center}
    { \hspace*{\fill} \\}
    
    
    \begin{verbatim}
<Figure size 1152x648 with 0 Axes>
    \end{verbatim}

    

    % Add a bibliography block to the postdoc
    
    
    
    \end{document}
